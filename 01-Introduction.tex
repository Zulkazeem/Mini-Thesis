\macronym{ABC}{A Better Computer}
\xnym[OPT-ARG]{ABCD}{Expanded CDE}


\chapter{Introduction}

\section{Background}
\label{sec:testbib}

The detection of lead and copper in water systems is not only crucial for public health but has gained renewed urgency with the
Environmental Protection Agency's mandate to replace all lead pipes within the next 10 years \cite{epa_lead_2023}. Lead contamination, in
particular, poses severe health risks, including developmental delays, neurological impairments, and cardiovascular issues, especially in children and pregnant women. Similarly, elevated levels of copper can cause gastrointestinal distress and other health complications.
Traditionally, detecting these metals in buried pipes has relied on invasive methods such as
physical excavation and chemical testing, which are labor-intensive, time-consuming, and disruptive to water supply systems. Moreover, these methods often require significant resources and may not provide real-time monitoring capabilities.

In response to these challenges, water authorities and researchers are actively exploring innovative, non-invasive detection methods that can accurately identify lead and copper pipes without the need for extensive excavation. This pursuit has led to advancements in simulation software and machine learning techniques, offering promising avenues for enhancing detection capabilities.

This study addresses these pressing needs by proposing a novel approach that integrates COMSOL Multiphysics simulations with machine learning algorithms. By combining the detailed environmental simulations provided by COMSOL with the analytical power of machine learning, this method aims to provide a robust and efficient solution for detecting lead and copper in buried pipes.

The integration of COMSOL simulations allows for the creation of realistic models that simulate the complex interactions between pipes, soil conditions, and other environmental factors. This includes adding a soil block to replicate the conditions in which pipes are buried, ensuring that the simulations closely mimic real-world scenarios. Furthermore, the use of accelerometers for data acquisition adds a layer of precision by capturing vibration and response data, which are critical indicators for detecting the presence of metals based on their distinct physical properties.

In parallel, machine learning techniques, particularly convolutional neural networks (CNNs), are employed to analyze the collected data and distinguish between lead and copper pipes. By training the CNN with a large dataset generated from COMSOL simulations, this study aims to enhance the accuracy and reliability of metal detection, thereby offering water authorities a practical tool for identifying and prioritizing pipe replacements.


\begin{figure}[t]
  \centering
  \includegraphics[%
  % REQUIRED alternate text for tagging.
  alt={The logo of EMSE-368/468.}%
  width=0.4\hsize]%
  {./Pipe.jpg}
  \caption{Logo of EMSE-368/468.}
  \label{fig:logo}
\end{figure}



\section{Motivation}
\label{sec:testfig}

Lead and copper contamination in drinking water is a significant public health concern \cite{epa_lead_2023}. Traditional detection methods involve physical inspections and chemical tests, which can be both time-consuming and disruptive. Recent advancements in simulation software and machine learning offer potential solutions to these challenges by providing non-invasive, accurate, and efficient detection methods.

Overall, this integrated approach not only addresses the immediate need for efficient metal detection in water systems but also sets the stage for future advancements in real-time monitoring and predictive maintenance strategies. As water utilities strive to comply with regulatory mandates and safeguard public health, innovative solutions that blend simulation technologies with advanced analytics are poised to play a pivotal role in transforming how lead and copper contamination are managed in infrastructure systems.


% Uncomment for bibliography on each chapter.
% \bibliographystyle{plainnat}				
% \markright{\textit{Bibliography}}
% \renewcommand{\chaptername}{}
% \bibliography{my_references}

% \vfill


%%% Local Variables:
%%% mode: latex
%%% TeX-master: "DMSE-Thesis"
%%% End:
